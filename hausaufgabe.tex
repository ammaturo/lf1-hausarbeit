\documentclass{ejsreport}
\addbibresource{lit.bib}
\usepackage{tabularx}
\usepackage{blindtext}

% ----------------------------------------------------------------------------

\begin{document}

\title{Hilfe zur Erziehung}
\author{
  one
  \and
  Anna Ammaturo
  \and
  three
  \and
  four
  \and
  five
  \and
  six
  \and
  seven  
}

\maketitle{}
\tableofcontents{}

\setcounter{chapter}{1}
\chapter[Rechte und Pflichten der Beteiligten]{Rechte und Pflichten der Beteiligten \chapterauthor{Anna Ammaturo}}

In Deutschland ist §1 des Sozialgesetzbuches~VIII (SGB~VIII) 
die Leitnorm der Kinder- und Jugendhilfe. \autocite[Vgl.][S.78]{Albrecht2019}
Junge Menschen haben das Recht 

% Zusätzlich muss das Leistungsangebot geeignet sein,
% den bestehenden Mangel zu beseitigen oder günstig
% zu beeinflussen.

% Wenn der junge Mensch z.B. ein Problem in der Schule hat,
% kann es sein, dass er nicht Hilfe zur Erziehung benötigt,
% sondern einfach eine Nachhilfe.

% Ausserdem muss geprüft werden, ob nicht eventuell die Eltern
% oder nahe Verwandte das Problem beheben können.


\section{Kindeswohl}

Kinder sind von Natur aus auf  
Laut dem Achten Buch des Sozialgesetzbuches (SGB VIII) besteht ein Rechtsanspruch 
auf Hilfe zur Erziehung, wenn eine dem Wohl des Kindes 
entsprechende Erziehung nicht gewährleistet ist 
(§~27 Absatz~1 SGB~VIII). 

Kindeswohl ist allerdings nicht genau definiert. 
Es muss jedesmal die Gesamtsituation des jungen
Menschen im Einzelfall betrachtet werden, um zu erkennen, 
ob das Kindeswohl gefährdet ist.
Die Kindeswohlgefährdung kann zum Beispiel 
eine der folgenden Formen annehmen: \autocite[Vgl.][S.~38~ff.]{Maihorn2009}

\begin{description}
  \item [Körperliche Misshandlung] 
  bewusste oder unbewusste Handlungen, die zu nicht zu\-fälligen
  körperlichen Schmerzen, Verletzungen oder sogar zum Tod führen.
  \item [Sexuelle Misshandlung]
  eine grenzüberschreitende sexuelle Handlung eines Erwachsenen
  oder Jugendlichen an einem Kind, wobei eine Machtposition
  ausgenutzt wird.
  \item [Vernachlässigung]
  wenn die Eltern unfähig sind, die körperlichen, seelischen, geistigen
  und materiellen Grundbedürfnisse des Kindes zu befriedigen. 
  Dazu gehören z.B. die Ernährung, Kleidung und die Gesundheitsfürsorge.
  \item [Psychische / emotionale Misshandlung]
  chronische unzureichende Handlungen und Beziehungsformen
  von Sorgeberechtigten zum Kind, die nicht dem Alter des Kindes
  angepasst sind. Dem Kind wird eingeredet, es sein wertlos, ungeliebt
  oder ungewollt.
  \item [Beeinträchtigung der elterlichen Erziehungskompetenz]
  wenn Eltern wegen einer psychischen Erkrankung, 
  geistigen Behinderung oder Substanzabhängigkeit eine eingeschränkte
  Erziehungskompetenz haben, die die Kinder beeinträchtigt.
\end{description}

\section{Rechte und Pflichten der Eltern}

Das Jugendamt berät die Eltern über die Möglichkeiten der Hilfe. 
Die Eltern haben das Recht, zwischen Einrichtungen und Diensten verschiedener 
Träger zu wählen und Wünsche zur Gestaltung der Hilfe zu äußern 
(§~5~Abs.~1 SGB~VIII). Allerdings müssen sich dabei die Kosten 
in einem vertretbaren Rahmen bewegen. (§~5~Abs.~2 SGB~VIII).
Ausserdem müssen die Eltern bei der Gestaltung eines Hilfeplans 
beteiligt werden (§~36~Abs.~1 SGB~VIII).

\section{Rechte und Pflichten der Kinder und Jugendlichen}

Kinder und Jugendliche können sich nach §~8 SGB~VIII in allen Angelegenheiten der Erziehung an das Jugendamt wenden.
haben Anspruch auf Beratung auch ohne Kenntnis der Eltern.
müssen an allen Entscheidungen beteiligt werden.
Müssen einbezogen werden, wenn die Einschätzung einer Kindeswohlgefährdung durchgeführt wird 
(§~8a SGB~VIII).
sind bei der Erstellung eines Hilfeplans umfassend beteiligt 
(§~36 SGB~VIII).

\section{Krisenintervention}

Oft kooperieren die Familien mit der Jugendhilfe. 
Allerdings gibt es auch Fälle, in denen die Familien nicht kooperieren. 
Das mag sein, weil sie den Bedarf nicht sehen oder das Stigma fürchten, 
auf Hilfe angewiesen zu sein.

In solchen Fällen kommt die Krisenintervention ins Spiel. 
Das Jugendamt kann jedoch nur intervenieren, 
wenn das Wohl des Kindes gefährdet ist. 
Wie schon weiter oben beschrieben wurde, 
muss das für jeden Einzelfall nachgewiesen werden. 
Das ist oft schwierig. 
Deswegen wird empfohlen, fachliche Verfahrensstandards einzuhalten, 
damit der Nachweis der Kindeswohlgefährdung 
besser durchgeführt werden kann.

Egal, ob eine Krisenintervention durchgeführt wird 
oder ob die Eltern kooperieren, der nächste Schritt ist die Hilfeplanung.

\printbibliography

\end{document}
