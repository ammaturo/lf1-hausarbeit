\documentclass{ejsreport}
\addbibresource{lit.bib}
\usepackage{tabularx}
\usepackage{blindtext}

% ----------------------------------------------------------------------------

\begin{document}
\setcounter{chapter}{1}
\chapter{Rechte und Pflichten der Beteiligten}
Anna Ammaturo
\\
\\
In Deutschland ist §1 des Sozialgesetzbuches VIII (SGB VIII) 
die Leitnorm der Kinder- und Jugendhilfe. \autocite[Vgl.][S.78]{Albrecht2019}
Junge Menschen haben das Recht 

\section{Kindeswohl}

Wenn man über die Rechten und Pflichten der Beteiligten 
an der Hilfe zur Erziehung spricht,
ist ein zentrales Konzept das Kindeswohl. \marginpar{Formulierung} 
Laut dem Achten Buch des Sozialgesetzbuches (SGB VIII) besteht ein Rechtsanspruch 
auf Hilfe zur Erziehung, wenn eine dem Wohl des Kindes 
entsprechende Erziehung nicht gewährleistet ist 
(§~27 Absatz~1 SGB~VIII). Kindeswohl ist allerdings nicht
genau definiert. Es muss jedesmal die Gesamtsituation des jungen
Menschen im Einzelfall betrachtet werden, um zu erkennen, 
ob das Kindeswohl gefährdet ist.

\subsection{Beispiele für Kindeswohlgefährdung}
Die Kindeswohlgefährdung kann zum Beispiel 
eine der folgenden Formen annehmen: \autocite[Vgl.][S.~38~ff.]{Maihorn2009}

\begin{description}
  \item [Körperliche Misshandlung] 
  bewusste oder unbewusste Handlungen, die zu nicht zu\-fälligen
  körperlichen Schmerzen, Verletzungen oder sogar zum Tod führen.
  \item [Sexuelle Misshandlung]
  eine grenzüberschreitende sexuelle Handlung eines Erwachsenen
  oder Jugendlichen an einem Kind, wobei eine Machtposition
  ausgenutzt wird.
  \item [Vernachlässigung]
  wenn die Eltern unfähig sind, die körperlichen, seelischen, geistigen
  und materiellen Grundbedürfnisse des Kindes zu befriedigen. 
  Dazu gehören z.B. die Ernährung, Kleidung und die Gesundheitsfürsorge.
  \item [Psychische / emotionale Misshandlung]
  chronische unzureichende Handlungen und Beziehungsformen
  von Sorgeberechtigten zum Kind, die nicht dem Alter des Kindes
  angepasst sind. Dem Kind wird eingeredet, es sein wertlos, ungeliebt
  oder ungewollt.
  \item [Beeinträchtigung der elterlichen Erziehungskompetenz]
  wenn Eltern wegen einer psychischen Erkrankung, 
  geistigen Behinderung oder Substanzabhängigkeit eine eingeschränkte
  Erziehungskompetenz haben, die die Kinder beeinträchtigt.
\end{description}

% Zusätzlich muss das Leistungsangebot geeignet sein,
% den bestehenden Mangel zu beseitigen oder günstig
% zu beeinflussen.

% Wenn der junge Mensch z.B. ein Problem in der Schule hat,
% kann es sein, dass er nicht Hilfe zur Erziehung benötigt,
% sondern einfach eine Nachhilfe.

% Ausserdem muss geprüft werden, ob nicht eventuell die Eltern
% oder nahe Verwandte das Problem beheben können.

\section{Rechte und Pflichten der Eltern}

Test \autocite[Vgl.][]{Albrecht2019}

\section{Rechte und Pflichten der Kinder}

\section{Krisenintervention}

\printbibliography


\end{document}
